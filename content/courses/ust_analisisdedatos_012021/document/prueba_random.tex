% Options for packages loaded elsewhere
\PassOptionsToPackage{unicode}{hyperref}
\PassOptionsToPackage{hyphens}{url}
\PassOptionsToPackage{dvipsnames,svgnames*,x11names*}{xcolor}
%
\documentclass[
  10pt,
  answers]{article}
\usepackage{lmodern}
\usepackage{amssymb,amsmath}
\newtheorem{problem}{Problema}
\usepackage{ifxetex,ifluatex}
\ifnum 0\ifxetex 1\fi\ifluatex 1\fi=0 % if pdftex
  \usepackage[T1]{fontenc}
  \usepackage[utf8]{inputenc}
  \usepackage{textcomp} % provide euro and other symbols
\else % if luatex or xetex
  \usepackage{unicode-math}
  \defaultfontfeatures{Scale=MatchLowercase}
  \defaultfontfeatures[\rmfamily]{Ligatures=TeX,Scale=1}
\fi
% Use upquote if available, for straight quotes in verbatim environments
\IfFileExists{upquote.sty}{\usepackage{upquote}}{}
\IfFileExists{microtype.sty}{% use microtype if available
  \usepackage[]{microtype}
  \UseMicrotypeSet[protrusion]{basicmath} % disable protrusion for tt fonts
}{}
\makeatletter
\@ifundefined{KOMAClassName}{% if non-KOMA class
  \IfFileExists{parskip.sty}{%
    \usepackage{parskip}
  }{% else
    \setlength{\parindent}{0pt}
    \setlength{\parskip}{6pt plus 2pt minus 1pt}}
}{% if KOMA class
  \KOMAoptions{parskip=half}}
\makeatother
\usepackage{xcolor}
\IfFileExists{xurl.sty}{\usepackage{xurl}}{} % add URL line breaks if available
\IfFileExists{bookmark.sty}{\usepackage{bookmark}}{\usepackage{hyperref}}
\hypersetup{
  pdftitle={ ICO 187 ANÁLISIS DE DATOS},
  pdfauthor={Examen},
  colorlinks=true,
  linkcolor=blue,
  filecolor=Maroon,
  citecolor=blue,
  urlcolor=blue,
  pdfcreator={LaTeX via pandoc}}
\urlstyle{same} % disable monospaced font for URLs
\usepackage[margin=0.75in]{geometry}
\usepackage{longtable,booktabs}
% Correct order of tables after \paragraph or \subparagraph
\usepackage{etoolbox}
\makeatletter
\patchcmd\longtable{\par}{\if@noskipsec\mbox{}\fi\par}{}{}
\makeatother
% Allow footnotes in longtable head/foot
\IfFileExists{footnotehyper.sty}{\usepackage{footnotehyper}}{\usepackage{footnote}}
\makesavenoteenv{longtable}
\usepackage{graphicx}
\makeatletter
\def\maxwidth{\ifdim\Gin@nat@width>\linewidth\linewidth\else\Gin@nat@width\fi}
\def\maxheight{\ifdim\Gin@nat@height>\textheight\textheight\else\Gin@nat@height\fi}
\makeatother
% Scale images if necessary, so that they will not overflow the page
% margins by default, and it is still possible to overwrite the defaults
% using explicit options in \includegraphics[width, height, ...]{}
\setkeys{Gin}{width=\maxwidth,height=\maxheight,keepaspectratio}
% Set default figure placement to htbp
\makeatletter
\def\fps@figure{htbp}
\makeatother
\setlength{\emergencystretch}{3em} % prevent overfull lines
\providecommand{\tightlist}{%
  \setlength{\itemsep}{0pt}\setlength{\parskip}{0pt}}
\setcounter{secnumdepth}{5}
\usepackage[spanish]{babel}
\usepackage{svg}
\usepackage{attachfile}
\usepackage[normalem]{ulem}
\usepackage{pdflscape}
\usepackage{multicol}% Just for this example
\usepackage{filecontents}
\usepackage{catchfile,environ,tikz}
\ifluatex
  \usepackage{selnolig}  % disable illegal ligatures
\fi

\title{\textsc{\Large{\includegraphics[width=2in,height=\textheight]{logo_fen.jpg}\\
ICO 187 ANÁLISIS DE DATOS}}\vspace{-2ex}}
\author{\textsc{\Large{Examen}}}
\date{}
% copper color
\definecolor{copper}{rgb}{0.72, 0.45, 0.2}
\begin{document}
\maketitle
\vspace{-0.75cm}
\begin{center}
\normalsize{\textbf{Año:} \textbf{2021}} \\
\normalsize{\textbf{Profesor:} \textbf{Sebastián Egaña}} \\
%\normalsize{\textbf{Ayudante:} \textit{}} \\
% \normalsize{Límite máximo de páginas: } \\
\vspace{0.25cm}
\end{center}


\begin{filecontents*}{bankA.tex}
\begin{questionblock}
Question 1
\end{questionblock}
\begin{questionblock}
Question 2
\end{questionblock}
\begin{questionblock}
Question 3
\end{questionblock}
\begin{questionblock}
Question 4
\end{questionblock}
\begin{questionblock}
Question 5
\end{questionblock}
\begin{questionblock}
Question 6
\end{questionblock}
\begin{questionblock}
Question 7
\end{questionblock}
\begin{questionblock}
Question 8
\end{questionblock}
\begin{questionblock}
Question 9
\end{questionblock}
\begin{questionblock}
Question 10
\end{questionblock}
\end{filecontents*}

\makeatletter\% Taken from \url{https://tex.stackexchange.com/q/109619/5764}
\def\declarenumlist#1#2#3{%
  \edef\csname pgfmath@randomlist@#1\endcsname{#3}%
  \count@\@ne
  \loop
    \edef
    \csname pgfmath@randomlist@#1@\the\count@\endcsname
      {\the\count@}
    \ifnum\count@<#3\relax
    \advance\count@\@ne
  \repeat}
\def\prunelist#1{%
  \xdef\csname pgfmath@randomlist@#1\endcsname
          {\the\numexpr\csname pgfmath@randomlist@#1\endcsname-1\relax}
  \count@\pgfmath@randomtemp 
  \loop
    \global\let
    \csname pgfmath@randomlist@#1@\the\count@\endcsname
    \csname pgfmath@randomlist@#1@\the\numexpr\count@+1\relax\endcsname
    \ifnum\count@<\csname pgfmath@randomlist@#1\endcsname\relax
      \advance\count@\@ne
  \repeat}
\makeatother

\% Define how each questionblock should be handled
\newcounter{questionblock}
\newcounter{totalquestions}
\NewEnviron{questionblock}{}\%

\newcommand{\randomquestionsfrombank}[2]{%
  \CatchFileDef{\bank}{#1}{}% Read the entire bank of questions into \bank
  \setcounter{totalquestions}{0}% Reset total questions counters  ***
  \RenewEnviron{questionblock}{\stepcounter{totalquestions}}% Count every question  ***
  \bank% Process file  ***
  \declarenumlist{uniquequestionlist}{1}{\thetotalquestions}% list from 1 to totalquestions inclusive.
  \setcounter{totalquestions}{#2}% Start the count-down
  \RenewEnviron{questionblock}{%
    \stepcounter{questionblock}% Next question
    \ifnum\value{questionblock}=\randomquestion 
      \par% Start new paragraph
      \BODY% Print question
    \fi
  }%
  \foreach \uNiQueQ in {1,...,#2} {% Extract #2 random questions
    \setcounter{questionblock}{0}% Start fresh with question block counter
    \pgfmathrandomitem\randomquestion{uniquequestionlist}% Grab random question from list
    \xdef\randomquestion{\randomquestion}% Make random question available globally
    \prunelist{uniquequestionlist}% Remove picked item from list
    \bank% Process file
  }}

\begin{multicols}{3}
  \foreach \x in {1,...,6} {
    \bigskip
    %
  \CatchFileDef{\bank}{bankA.tex}{}% Read the entire bank of questions into \bank
  \setcounter{totalquestions}{0}% Reset total questions counters  ***
  \RenewEnviron{questionblock}{\stepcounter{totalquestions}}% Count every question  ***
  \bank% Process file  ***
  %
  \edef\csname pgfmath@randomlist@uniquequestionlist\endcsname{\thetotalquestions}%
  \count@\@ne
  \loop
    \edef
    \csname pgfmath@randomlist@uniquequestionlist@\the\count@\endcsname
      {\the\count@}
    \ifnum\count@<\thetotalquestions\relax
    \advance\count@\@ne
  \repeat% list from 1 to totalquestions inclusive.
  \setcounter{totalquestions}{6}% Start the count-down
  \RenewEnviron{questionblock}{%
    \stepcounter{questionblock}% Next question
    \ifnum\value{questionblock}=\randomquestion 
      \par% Start new paragraph
      \BODY% Print question
    \fi
  }%
  \foreach \uNiQueQ in {1,...,6} {% Extract #2 random questions
    \setcounter{questionblock}{0}% Start fresh with question block counter
    \pgfmathrandomitem\randomquestion{uniquequestionlist}% Grab random question from list
    \xdef\randomquestion{\randomquestion}% Make random question available globally
    %
  \xdef\csname pgfmath@randomlist@uniquequestionlist\endcsname
          {\the\numexpr\csname pgfmath@randomlist@uniquequestionlist\endcsname-1\relax}
  \count@\pgfmath@randomtemp 
  \loop
    \global\let
    \csname pgfmath@randomlist@uniquequestionlist@\the\count@\endcsname
    \csname pgfmath@randomlist@uniquequestionlist@\the\numexpr\count@+1\relax\endcsname
    \ifnum\count@<\csname pgfmath@randomlist@uniquequestionlist\endcsname\relax
      \advance\count@\@ne
  \repeat% Remove picked item from list
    \bank% Process file
  }
  }
\end{multicols}

\end{document}
