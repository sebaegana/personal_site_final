\documentclass[12 pt,letterpaper]{article} 

\usepackage[bottom]{footmisc}
\usepackage[utf8]{inputenc}
\usepackage{float}
\usepackage{array}
\usepackage[nottoc]{tocbibind}
\usepackage{caption}
\usepackage{amsmath}
\usepackage[spanish]{babel}
\usepackage{setspace}
\usepackage{fancyhdr}
\usepackage{booktabs}
\usepackage{tabularx}
\usepackage{multirow,hhline}
\usepackage[T1]{fontenc}
\usepackage{lmodern}
\usepackage{amssymb,amsthm}
\usepackage{verbatim} 
\usepackage{comment}
\usepackage{enumitem}
\usepackage{amssymb}
\usepackage{tensor}
\usepackage{lscape}
\usepackage{longtable}

\usepackage[usenames,dvipsnames,svgnames,table]{xcolor}
\usepackage{tabularx}
\usepackage{array}
\usepackage{natbib}
\usepackage{delarray}
\usepackage{dcolumn}
\usepackage{float}
\usepackage{amsmath}
\usepackage{psfrag}
\usepackage{multirow,hhline}
\usepackage{amstext}
\usepackage{pstricks}
\usepackage{pst-plot}
\usepackage{amssymb}
\usepackage{fancyhdr}
\usepackage{dsfont}
\usepackage{transparent}

\usepackage{fontawesome}
\DeclareFontFamily{U}{fontawesomeOne}{}
\DeclareFontShape{U}{fontawesomeOne}{m}{n}
{<-> FontAwesome--fontawesomeone}{}
\DeclareRobustCommand\FAone{\fontencoding{U}\fontfamily{fontawesomeOne}\selectfont} 

\usepackage{framed}
\usepackage[framemethod=tikz]{mdframed}

\usepackage[top=1.5in,bottom=1in,right=1in,left=1in,headheight=65pt]{geometry}

\usepackage[colorlinks=true, linkcolor=magenta]{hyperref}
\hypersetup{
	colorlinks,
	citecolor=Red,
	linkcolor=Cyan,
	urlcolor=Red}


\newif\ifprint
\printtrue  % Para hacer que se convierta en pauta
% \begin{solution} y \end{solution} para generar respuestas


\ifprint
\newcommand{\tipo}{Pauta}
\newenvironment{solution}
{\begin{mdframed} \textbf{Respuesta:} \ \\}
	{\end{mdframed}}

\else
\newcommand{\tipo}{Enunciado}
\excludecomment{solution}
\fi	

%_____________________________________________________________________________________________________________________



%\usepackage{Sweave}
\begin{document}
%\input{ayudantia_N02_pauta-concordance}

	
	\pagestyle{fancy}
	\fancyhf{}
	\lhead{\center {\transparent{0.4}\includegraphics[width=3cm]{logo_fen.jpg}}}
	\rfoot{Página \thepage}
	\renewcommand{\headrulewidth}{0pt}
	\renewcommand{\footrulewidth}{0pt}	
	
\begin{center}
		
	\bigbreak
	\textbf{ICO 187 ANÁLISIS DE DATOS}\\
	\small{2021}
	\break
	\textbf{CLASE 03: Comprensión de datos numéricos y no numéricos}\\
\end{center}

\begin{flushright}	
		
	Profesor: \href{mailto:sebastianeganasa@santotomas.cl}{Sebastián Egaña}

\end{flushright}

\section{Repaso de la clase pasada}

\begin{itemize}
	\item Referencia relativa: Guarda relación con la columna y fila en donde se encuentran, y son editadas al momento de ser copiadas.
	
	\item Referencia absoluta: La relación con la fila y columna es permanente, considerando la fijación de estas. Se realiza a través de la utilización de nombres, como también a través de la utilización del signo \$.
\end{itemize}
	

\section{Data Númerica}

Recapitulando lo visto en la clase pasada, una primera clasificación para las variables cuantitativas corresponde a la siguiente: 

\begin{itemize}
	\item Continua
	\item Discreta
\end{itemize}

A pesar de esto, nos interesa un enfoque de datos, según esto la relevancia se relaciona con la utilización de capacidad de almacenamiento de cada una de las variables. Por lo tanto, podemos generar la siguiente clasificación en base a RStudio: 

\begin{itemize}
	\item Numeric: Corresponde a valores numéricos de cualquier largo, que incluso pueden contener números decimales. 
	\item Integer: Corresponde a valores que solo pueden ser números enteros de cualquier largo. 
	\item Complex: Corresponde a valores complejos, como del tipo raíces negativas, etc. 
\end{itemize}

En otro caso, pueden ocurrir casos en donde se este en presencias de variables que se almacenan como números, pero no son realmente números. Por ejemplo un RUT 12.345.678-9 podría ser almacenada como 123456789 como si fuera ciento veinte y tres millones cuatrocientos y algo. En este caso, se debe especificar que la variable corresponde a una variable númerica, pero es un identificador. Esto también puede pasar en el caso de variables binomiales que son generadas a través de 1 y 0. \\

Veamos ahora el detalle en el orden de precedencia de las operatorias:

\begin{enumerate}
	\item Evalua los parentesis.
	\item Evalua los rangos.
	\item Evalua las intersecciones (espacio).
	\item Evalua las uniones (;).
	\item Negaciones (-).
	\item Convertir porcentajes.
	\item Exponenciales.
	\item Multiplicación y división, de igual manera (* y /).
	\item Adición y substracción, de igual manera (+ y -).
	\item Operadores de texto (\&).
	\item Comparativos (=, <>, <=, >=).
	
	En el caso de encontrarse elementos en el mismo orden, se evaluan de izquierda a derecha. 
\end{enumerate}

\subsection{Intersecciones y uniones en Excel}

Veamos el siguiente video: \href{https://web.microsoftstream.com/video/a00c02ae-3403-40f1-92ee-a5f5e49843f2}{Enlace acá}


\subsection{Funciones que nos pueden servir}

\begin{itemize}
	\item ESNUMERO: Valor se refiere a un número. \href{https://support.microsoft.com/es-es/office/funciones-es-0f2d7971-6019-40a0-a171-f2d869135665}{Enlace acá}

\end{itemize}

\noindent
{\Huge \faBell} Ejercicio: Desarrolle una función que retorne 1 si la celda contiene un número, y 0 si no lo contine. 

\begin{solution}
	
Considere la siguiente función \\

=SI(ESNUMERO(A1); 1; 2)
	
\end{solution}

\section{Arreglos en Excel}

Desde 2018, Excel incorpora la opción de funciones que se aplican en rangos.

\begin{itemize}
	\item 	Por ejemplo, la función SECUENCIA \href{https://support.microsoft.com/es-es/office/función-secuencia-57467a98-57e0-4817-9f14-2eb78519ca90}{Enlace acá} 
\end{itemize}

\noindent
{\Huge \faBell} Ejercicio: Aplique un arreglo que replique la función SUMAR.SI.

\begin{solution}
Considere la aplicación de las siguientes funciones:\\

=SUMA(SI(\$C\$5:\$C\$11=\$C19;D\$5:D\$11;0))
\end{solution}

\section{Funciones comunes}

\begin{itemize}
	\item SUMA
	\item MIN
	\item MAX
	\item AVERAGE
	\item CONTAR
	\item LARGO
	\item SUMAR.SI
	\item PROMEDIO.SI
	\item CONTAR.SI
	\item DIAS
	\item HOY
\end{itemize}

\section{Funciones matemáticas}

\begin{itemize}
	\item ESNUMERO
	\item ALEATORIO
	\item REDONDEAR
	\item MEDIANA
	\item PI
	\item POTENCIA
	\item RESIDUO
	\item NUMERO.ROMANO
\end{itemize}

\noindent
{\Huge\faComment} Pregunta: La función RESIDUO, por lo general puede ser utilizada para determinar si un número par o no, ¿por qué?

\section{Tener en cuenta siempre}

\begin{itemize}
	\item Los valores vacíos, Excel los asume como 0. Ver operaciones. 
	\item Revisar siempre el formato de las fechas.
	\item Revisar siempre el formato númerico; marcador de decimales versus marcador de miles. 
	\item Revisar cuál es el marcador para los parámetros de las funciones. 
\end{itemize}

\section{Desafío \faWarning}

\begin{itemize}
	\item Genere una función que retorno el primer valor no blanco (vacío de un vector). 
	
	Considere la siguiente tabla:
	
	\begin{center}
	\begin{tabular}{|c|c|c|c|}
		\hline
		A & B & C & D \\
		\hline
		100 &  & 100 &  \\
		\hline
		AAA &  &  & AAA \\
		\hline
	\end{tabular}
	\end{center}
	
	En donde la función debería ir en la columna A, identificando el primer valor no blaco y retornarlo en la misma columna. 
	
\end{itemize}

\section{Próxima Clase}

\begin{itemize}
	\item Control Individual de Participación (5\% de ponderación).
\end{itemize}

\section{Aparte}

\begin{itemize}
	\item Ver la disponibilidad de Power Pivot en Excel: \href{https://support.microsoft.com/es-es/office/iniciar-el-complemento-power-pivot-para-excel-a891a66d-36e3-43fc-81e8-fc4798f39ea8}{Enlace acá}	
	
	\item Instalar Power BI en Español: \href{https://powerbi.microsoft.com/es-es/desktop/}{Enlace acá}
	
	\item Crearse una cuenta en R-Studio Cloud: \href{https://rstudio.cloud/}{Enlace acá}
		
\end{itemize}

\section{Fechas relevantes}

\begin{tabular}{|c|l|c|c|}
	\hline
	Unidad & Evaluación & Ponderación & Fecha \\
	\hline
	Unidad I & Evaluación diagnóstica & & 25/03/2021 \\
	\hline
	& Evaluación Individual Participación & (5\%) & 01/04/2021 \\
	\hline
	& Evaluación Grupal & (15\%) & 27/04/2021 \\
	\hline
	& Evaluación Individual - Sumativa I & (30\%) & 11/05/2021 \\
	\hline
	Unidad II & Evaluación Formativa & & 13/05/2021 \\
	\hline
	& Evaluación Individual Participación & (5\%) & 27/05/2021 \\
	\hline
	& Evaluación Grupal & (10\%) & 08/06/2021 \\
	\hline
	& Evaluación Individual - Sumativa II & (15\%) & 17/06/2021 \\
	\hline
	Unidad III & Evaluación Formativa & & 22/06/2021 \\
	\hline
	& Evaluación Individual Participación & (5\%) & 29/06/2021 \\
	\hline
	& Evaluación Individual Sesión I- Sumativa III & (15\%) & 08/07/2021 \\
	\hline
	& Evaluación Individual Sesión II- Sumativa III & (15\%) & 13/07/2021 \\
	\hline
\end{tabular}

\end{document}



