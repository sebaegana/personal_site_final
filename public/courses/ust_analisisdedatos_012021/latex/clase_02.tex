\documentclass[12 pt,letterpaper]{article} 

\usepackage[bottom]{footmisc}
\usepackage[utf8]{inputenc}
\usepackage{float}
\usepackage{array}
\usepackage[nottoc]{tocbibind}
\usepackage{caption}
\usepackage{amsmath}
\usepackage[spanish]{babel}
\usepackage{setspace}
\usepackage{fancyhdr}
\usepackage{booktabs}
\usepackage{tabularx}
\usepackage{multirow,hhline}
\usepackage[T1]{fontenc}
\usepackage{lmodern}
\usepackage{amssymb,amsthm}
\usepackage{verbatim} 
\usepackage{comment}
\usepackage{enumitem}
\usepackage{amssymb}
\usepackage{tensor}
\usepackage{lscape}
\usepackage{longtable}

\usepackage[usenames,dvipsnames,svgnames,table]{xcolor}
\usepackage{tabularx}
\usepackage{array}
\usepackage{natbib}
\usepackage{delarray}
\usepackage{dcolumn}
\usepackage{float}
\usepackage{amsmath}
\usepackage{psfrag}
\usepackage{multirow,hhline}
\usepackage{amstext}
\usepackage{pstricks}
\usepackage{pst-plot}
\usepackage{amssymb}
\usepackage{fancyhdr}
\usepackage{dsfont}
\usepackage{transparent}

\usepackage{framed}
\usepackage[framemethod=tikz]{mdframed}

\usepackage[top=1.5in,bottom=1in,right=1in,left=1in,headheight=65pt]{geometry}

\usepackage[colorlinks=true, linkcolor=magenta]{hyperref}
\hypersetup{
	colorlinks,
	citecolor=Red,
	linkcolor=Cyan,
	urlcolor=Red}


\newif\ifprint
\printtrue  % Para hacer que se convierta en pauta
% \begin{solution} y \end{solution} para generar respuestas


\ifprint
\newcommand{\tipo}{Pauta}
\newenvironment{solution}
{\begin{mdframed} \textbf{Respuesta:} \ \\}
	{\end{mdframed}}

\else
\newcommand{\tipo}{Enunciado}
\excludecomment{solution}
\fi	

%_____________________________________________________________________________________________________________________



%\usepackage{Sweave}
\begin{document}
%\input{ayudantia_N02_pauta-concordance}

	
	\pagestyle{fancy}
	\fancyhf{}
	\lhead{\center {\transparent{0.4}\includegraphics[width=3cm]{logo_fen.jpg}}}
	\rfoot{Página \thepage}
	\renewcommand{\headrulewidth}{0pt}
	\renewcommand{\footrulewidth}{0pt}	
	
\begin{center}
		
	\bigbreak
	\textbf{ICO 187 ANÁLISIS DE DATOS}\\
	\small{2021}
	\break
	\textbf{CLASE 02: Aspectos básicos de la planilla de cálculo}\\
\end{center}

\begin{flushright}	
		
	Profesor: \href{mailto:sebastianeganasa@santotomas.cl}{Sebastián Egaña}

\end{flushright}

\section{Hoja de cálculo}

\begin{itemize}
	\item Barras: Barra de menús, barra de opciones, barra de formulas. 
	
	\item Hojas de cálculo: Copiar hojas, nombrar hojas, traspasar hojas entre distintos archivos de Excel, etc. 
	
	\item Filas: Las filas son denominadas por números y corresponden a un dato en la base de datos. 
	
	\item Columnas: Las columnas son denominadas por letras y corresponden a una variable en la base de datos. 
	
\end{itemize}

\section{Tipos de errores y valores vacios}

\begin{itemize}
	\item Errores en formulas: por lo general va acompañado de un aviso, o de algún indicador. 

	\item Error en los datos: 
	
	\begin{itemize}
		
		\item \#\#\#\#\#: El ancho de la columna no es suficiente.
		
		\item \#¡NUM!: Operador incorrecto, o valores númericos no validos en la fórmula o función. 
		
		\item \#¡DIV/0!: División por cero.
		
		\item ¿NOMBRE?: Excel no reconoce la fórmula.
		
		\item \#N/A: Valor no disponible para la función o fórmula.
		
		\item \#¡REF!: referencia no valida. 
		
		\item \#¡NULO!: Cuando se específica una intersección no existente. 
		
	\end{itemize}
		

\end{itemize}

\section{Tipos de datos}

Primero hablarémos en general de los tipos de datos:

\begin{itemize}
	
	\item Datos númericos: En Excel, esto puede ocupar varios formatos pero la base siempre es un número. En el caso de softwares distintos de Excel, los datos númericos si pueden ser distintos: En Python existen Integers (enteros), Floating (racionales, pero más refiere a la inclusión del valor decimal), Complex Numbers (números complejos). En R ocurre algo parecido: existen Integers (enteros), Numeric (números en general) y Complex (números complejos).
	
	\item Datos no númericos: Esto corresponde en Excel al formato texto, que es visto más en terminos de palabras dando la posibilidad de operar en términos de cadena de caracteres. Para el caso de Python existen los Strings (cadena de caracteres) y Boolean (lógico, TRUE o FALSE). En R se le llaman Logical (lógico, TRUE o FALSE, pero convertible a número) y character (cadena de caracteres).
	
	\item Datos agrupados: Técnicamente esto no es un tipo de dato, pero es necesario hacer referencia a las agrupaciones. En el caso de Excel, se pueden generar matrices y vectores para realizar operaciones númericas. En cambio, en el caso de R se enriquece este concepto: existen List (listas), Array (arreglos), Data frames (tablas), Tibbles (tablas enriquecidas), etc.	
	
\end{itemize}

Ahora, en relación al caso particular de Excel:
	
\begin{itemize}

	\item Valores constantes: Puede ser un número, una fecha o un texto.
	
	\item Fórmulas: Es una secuencia formada por valores contantes, otras celdas, nombres de funciones u operadores. 
	
	\item Nombrar constantes

\end{itemize}

En relación al análisis de datos, no corresponden estrictamente a un tipo de dato, pero por las particularidades de Excel debe especificarse su diferencia. 

\subsection{Un caso particular:}

Toda fecha es un número, que parte desde el 1/1/1900 siendo esta fecha la número 1 y así en adelante. Según esto, la fecha no es más que un formato para dicho número, relacionado con la forma en que quiere ser presentada dicha fecha. En el caso de fechas y horarios, el horario viene representado por un decimal. Veamos un ejemplo:
	
Si se quiere expresar en decimales las 8:30 AM.
	
\[ 8*(1/24)+(30*(1/1440))=0.354167 \]
	
Esto es aplicable para cualquier hora, cambiado solo los valores relacionados con la conversión (en este caso 8 y 30).
	
\section{Coherencia en los datos}

Como se dijo anteriormente, en las filas van observaciones y en las columnas van variables. Dentro de cada variable debe haber una coherencia, existiendo el mismo tipo de dato dentro de ella. En relación a las filas, se debe intentar no añadir datos que tengan información incompleta o valores perdidos. 

\section{Atajos}

Veamos algunos atajos:

\begin{center}
	\begin{tabular}{|l|c|}
		\hline
		Moverse entre celdas & Flechas o Keyboard númerico \\
		\hline
		Pantalla Abajo & AVPAG \\
		\hline
		Pantalla Arriba & REPAG \\
		\hline
		Celda A1 & CTRL + INICIO \\
		\hline
		Filas y columnas ya escritos & CTRL + Flechas \\
		\hline
		Seleccionar filas o columnas & CTRL + SHIFT + Flechas \\
		\hline
		Eliminar Columnas & CTRL + - \\
		\hline
	\end{tabular}
\end{center}

\section{Formulas}

Se puede ingresar utilizando el signo = o +. Por lo general, se utilizar la barra de opciones para ingresarlas, pero existe una manera más avanzada de hacer esto. 

Para esto, debemos utilizar el TAB.

\section{Aparte}

\begin{itemize}
	\item Ver la disponibilidad de Power Pivot en Excel: \href{https://support.microsoft.com/es-es/office/iniciar-el-complemento-power-pivot-para-excel-a891a66d-36e3-43fc-81e8-fc4798f39ea8}{Enlace acá}	
	
	\item Instalar Power BI en Español: \href{https://powerbi.microsoft.com/es-es/desktop/}{Enlace acá}
	
	\item Crearse una cuenta en R-Studio Cloud: \href{https://rstudio.cloud/}{Enlace acá}
		
\end{itemize}









\end{document}



