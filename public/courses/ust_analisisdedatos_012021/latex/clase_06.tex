\documentclass[12 pt,letterpaper]{article} 

\usepackage[bottom]{footmisc}
\usepackage[utf8]{inputenc}
\usepackage{float}
\usepackage{array}
\usepackage[nottoc]{tocbibind}
\usepackage{caption}
\usepackage{amsmath}
\usepackage[spanish]{babel}
\usepackage{setspace}
\usepackage{fancyhdr}
\usepackage{booktabs}
\usepackage{tabularx}
\usepackage{multirow,hhline}
\usepackage[T1]{fontenc}
\usepackage{lmodern}
\usepackage{amssymb,amsthm}
\usepackage{verbatim} 
\usepackage{comment}
\usepackage{enumitem}
\usepackage{amssymb}
\usepackage{tensor}
\usepackage{lscape}
\usepackage{longtable}

\usepackage[usenames,dvipsnames,svgnames,table]{xcolor}
\usepackage{tabularx}
\usepackage{array}
\usepackage{natbib}
\usepackage{delarray}
\usepackage{dcolumn}
\usepackage{float}
\usepackage{amsmath}
\usepackage{psfrag}
\usepackage{multirow,hhline}
\usepackage{amstext}
\usepackage{pstricks}
\usepackage{pst-plot}
\usepackage{amssymb}
\usepackage{fancyhdr}
\usepackage{dsfont}
\usepackage{transparent}

\usepackage{fontawesome}
\DeclareFontFamily{U}{fontawesomeOne}{}
\DeclareFontShape{U}{fontawesomeOne}{m}{n}
{<-> FontAwesome--fontawesomeone}{}
\DeclareRobustCommand\FAone{\fontencoding{U}\fontfamily{fontawesomeOne}\selectfont} 

\usepackage{framed}
\usepackage[framemethod=tikz]{mdframed}

\usepackage[top=1.5in,bottom=1in,right=1in,left=1in,headheight=65pt]{geometry}

\usepackage[colorlinks=true, linkcolor=magenta]{hyperref}
\hypersetup{
	colorlinks,
	citecolor=Red,
	linkcolor=Cyan,
	urlcolor=Red}


\newif\ifprint
\printtrue  % Para hacer que se convierta en pauta
% \begin{solution} y \end{solution} para generar respuestas


\ifprint
\newcommand{\tipo}{Pauta}
\newenvironment{solution}
{\begin{mdframed} \textbf{Respuesta:} \ \\}
	{\end{mdframed}}

\else
\newcommand{\tipo}{Enunciado}
\excludecomment{solution}
\fi	

%_____________________________________________________________________________________________________________________



%\usepackage{Sweave}
\begin{document}
%\input{ayudantia_N02_pauta-concordance}

	
	\pagestyle{fancy}
	\fancyhf{}
	\lhead{\center {\transparent{0.4}\includegraphics[width=3cm]{logo_fen.jpg}}}
	\rfoot{Página \thepage}
	\renewcommand{\headrulewidth}{0pt}
	\renewcommand{\footrulewidth}{0pt}	
	
\begin{center}
		
	\bigbreak
	\textbf{ICO 187 ANÁLISIS DE DATOS}\\
	\small{2021}
	\break
	\textbf{CLASE 06: Análisis de datos}\\
\end{center}

\begin{flushright}	
		
	Profesor: \href{mailto:sebastianeganasa@santotomas.cl}{Sebastián Egaña}

\end{flushright}

\section{Funciones de bases de datos}

\begin{itemize}
	\item Por lo general, la más conocida corresponde a BUSCARV. La que nos permite buscar de manera vertical, teniendo su equivalente BUSCARH, que nos permite movernos en términos horizontales. Por último, existe una función XBUSCAR, que permite realizar búsquedas más complejas.

	\item Veamos como funciona esto. 	
\end{itemize}
	
\section{Desafío clase pasada \faWarning}

\begin{itemize}
	\item Genere una función que retorno el primer valor no blanco (vacío de un vector).
	
	\item Para esto, considerar las siguientes funciones:
	
	\begin{itemize}
		\item INDICE
		\item COINCIDIR
		\item ESBLANCO
	\end{itemize}

	\item Dichas funciones, operan parecido a las funciones BUSCAR antes vistas. Veamos algunos ejemplos antes de resolver el ejercicio. 
	
	\item Resolvamos el ejercicio ahora.
	
\end{itemize}

\begin{solution}
	Se debe aplicar funciones de la siguiente manera:\\
	
	=INDICE(C4:N4;1;COINCIDIR(1;INDICE(1-ESBLANCO(C4:N4);1;0);0)) \\
	
	Veamos esto parte por parte.
\end{solution} 

\section{Wildcards en Excel}

\begin{itemize}
	\item Refiere a carácteres que nos permiten hacer búsquedas de manera general dentro de una cadena de carácteres. 
	
	\item Existe tres símbolos relacionados con comodines en Excel: *, ? y \~.
\end{itemize}

\section{Funciones Lógicas}

\begin{itemize}
	\item Por lo general se utiliza la función SI como parte de esta categoría, pero también existen la función O, Y, NO y otras relacionadas con al función SI. Veamos la página de ayuda de Excel:  \href{https://support.microsoft.com/es-es/office/funciones-lógicas-referencia-e093c192-278b-43f6-8c3a-b6ce299931f5}{Click acá}
	
	\item Una finalidad común a estas funciones, es la generación de nuevas variables en la base de datos. 
	
	\item Veamos algunos ejemplos. 
\end{itemize}

\section{Próxima evaluación}

\begin{itemize}
	\item Generar grupos de cuatro personas (sin excepciones).
	
	\item Fecha 27/04/2021 (3 semanas más aprox.).
	
	\item Se entregará una tarea en dicha semana, y tendrán hasta el 04/05/2021 para entregarla. 
\end{itemize}

\section{Fechas relevantes}

\begin{center}
\begin{tabular}{|c|l|c|c|}
	\hline
	Unidad & Evaluación & Ponderación & Fecha \\
	\hline
	Unidad I & Evaluación diagnóstica & & 25/03/2021 \\
	\hline
	& Evaluación Individual Participación & (5\%) & 05/04/2021 \\
	\hline
	& Evaluación Grupal & (15\%) & 27/04/2021 - 04/05/2021 \\
	\hline
	& Evaluación Individual - Sumativa I & (30\%) & 11/05/2021 \\
	\hline
	Unidad II & Evaluación Formativa & & 13/05/2021 \\
	\hline
	& Evaluación Individual Participación & (5\%) & 27/05/2021 \\
	\hline
	& Evaluación Grupal & (10\%) & 08/06/2021 - 15/06/2021 \\
	\hline
	& Evaluación Individual - Sumativa II & (15\%) & 17/06/2021 \\
	\hline
	Unidad III & Evaluación Formativa & & 22/06/2021 \\
	\hline
	& Evaluación Individual Participación & (5\%) & 24/06/2021 \\
	\hline
	& Evaluación Individual Sesión I- Sumativa III & (15\%) & 08/07/2021 \\
	\hline
	& Evaluación Individual Sesión II- Sumativa III & (15\%) & 13/07/2021 \\
	\hline
\end{tabular}
\end{center}

\end{document}



