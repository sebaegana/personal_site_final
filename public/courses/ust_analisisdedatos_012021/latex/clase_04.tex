\documentclass[12 pt,letterpaper]{article} 

\usepackage[bottom]{footmisc}
\usepackage[utf8]{inputenc}
\usepackage{float}
\usepackage{array}
\usepackage[nottoc]{tocbibind}
\usepackage{caption}
\usepackage{amsmath}
\usepackage[spanish]{babel}
\usepackage{setspace}
\usepackage{fancyhdr}
\usepackage{booktabs}
\usepackage{tabularx}
\usepackage{multirow,hhline}
\usepackage[T1]{fontenc}
\usepackage{lmodern}
\usepackage{amssymb,amsthm}
\usepackage{verbatim} 
\usepackage{comment}
\usepackage{enumitem}
\usepackage{amssymb}
\usepackage{tensor}
\usepackage{lscape}
\usepackage{longtable}

\usepackage[usenames,dvipsnames,svgnames,table]{xcolor}
\usepackage{tabularx}
\usepackage{array}
\usepackage{natbib}
\usepackage{delarray}
\usepackage{dcolumn}
\usepackage{float}
\usepackage{amsmath}
\usepackage{psfrag}
\usepackage{multirow,hhline}
\usepackage{amstext}
\usepackage{pstricks}
\usepackage{pst-plot}
\usepackage{amssymb}
\usepackage{fancyhdr}
\usepackage{dsfont}
\usepackage{transparent}

\usepackage{fontawesome}
\DeclareFontFamily{U}{fontawesomeOne}{}
\DeclareFontShape{U}{fontawesomeOne}{m}{n}
{<-> FontAwesome--fontawesomeone}{}
\DeclareRobustCommand\FAone{\fontencoding{U}\fontfamily{fontawesomeOne}\selectfont} 

\usepackage{framed}
\usepackage[framemethod=tikz]{mdframed}

\usepackage[top=1.5in,bottom=1in,right=1in,left=1in,headheight=65pt]{geometry}

\usepackage[colorlinks=true, linkcolor=magenta]{hyperref}
\hypersetup{
	colorlinks,
	citecolor=Red,
	linkcolor=Cyan,
	urlcolor=Red}


\newif\ifprint
\printtrue  % Para hacer que se convierta en pauta
% \begin{solution} y \end{solution} para generar respuestas


\ifprint
\newcommand{\tipo}{Pauta}
\newenvironment{solution}
{\begin{mdframed} \textbf{Respuesta:} \ \\}
	{\end{mdframed}}

\else
\newcommand{\tipo}{Enunciado}
\excludecomment{solution}
\fi	

%_____________________________________________________________________________________________________________________



%\usepackage{Sweave}
\begin{document}
%\input{ayudantia_N02_pauta-concordance}

	
	\pagestyle{fancy}
	\fancyhf{}
	\lhead{\center {\transparent{0.4}\includegraphics[width=3cm]{logo_fen.jpg}}}
	\rfoot{Página \thepage}
	\renewcommand{\headrulewidth}{0pt}
	\renewcommand{\footrulewidth}{0pt}	
	
\begin{center}
		
	\bigbreak
	\textbf{ICO 187 ANÁLISIS DE DATOS}\\
	\small{2021}
	\break
	\textbf{CLASE 04: Manipulación de datos numéricos y no numéricos}\\
\end{center}

\begin{flushright}	
		
	Profesor: \href{mailto:sebastianeganasa@santotomas.cl}{Sebastián Egaña}

\end{flushright}

\section{Diagnóstico}

\begin{itemize}
	\item Excel intermedio.
	
	\item Conocimientos de programación bajo.
	
	\item Alta disposición por aprender a programar.
	
	\item Experiencia en tablas dinámicas.
	
	\item Nadie sabe lo que es Solver.
	
	\item xls vs xlsx.	
\end{itemize}

\section{Data No Númerica}

Recordar, que para el entorno de programación y análisis de datos, variable refiere a un elemento dentro del entorno de programación utilizado. Por otra parte, el término variable utilizado en estadística se corresponde con los tipos de datos en programación. 

La data no númerica, se puede categorizar en el enfoque estadístico dentro del grupo de variables cualitativas:

\begin{enumerate}
	\item Nominal
	\item Ordinal
	\item Binaria
\end{enumerate}

Donde la variable nominal refiere a las variables que no posee un orden interno, a diferencia de la variable ordinal. En el caso de la variable binaria, corresponde a categorías dicotómicas o contrapuestas y vimos que puede ser representada de forma númerica. 

Desde el enfoque de datos, en base al lenguaje R, existen los siguientes tipos de datos no númericos:

\begin{enumerate}
	\item Character: Corresponde a cadenas de carácteres en general. 
	\item Logical: Refiere a un tipo de dato lógico valores TRUE o FALSE. Se conoce también como booleanos. 
\end{enumerate}

Como ya se analizó, las fechas corresponden a un caso particular: de manera interna, se almacenan como números pero se muestran en la forma correspondiente una fecha. En este sentido, son muy parecidas a un texto. Para el caso de R, el formato de fecha más utilizado corresponde a POSIXct y Date, en donde el primero posee la cualidad de poder almacenar fecha y hora. 

Para este tipo de datos, lo más importante para este tipo de datos corresponde a la extracción y transformación de los mismos. Ya revisamos los tipos de datos, pero en este caso la extracción y transformación de datos no se limita solo a los datos que son cadenas de carácteres; las fechas y algunos formatos de número también pueden ser tratados como cadenas de carácteres debido a la forma en como trabaja Excel. Veamos un ejemplo de esto:\\

\noindent
{\Huge \faBell} Ejercicio: Descarguemos algunos datos de la siguiente página  \href{https://es.investing.com/indices/investing.com-btc-usd-historical-data}{Enlace acá} e intenemos reconstruir la fecha de cada transacción en la bolsa.

\begin{solution}
	Podemos utilizar las siguientes funciones: FECHA, DIA, MES y AÑO, IZQUIERDA, DERECHA y EXTRAE. \\
	
	=FECHA(DERECHA(A2;4);EXTRAE(A2;4;2);IZQUIERDA(A2;2)) \\
	
	Otra opción, podría ser usar REEMPLAZAR, y FECHA NUMERO. \\
	
	=REEMPLAZAR(A2;3;1;"/") \\
	
	luego =REEMPLAZAR(C2;6;1;"/") \\
	
	y por último =FECHANUMERO(D2)	
\end{solution}

\subsection{Aplicación en relación con los contenidos de la clase pasada}

Utilizando la siguiente clase\_04 fifa.csv, realicemos un ejercicio relacionado con la función SUMAR.SI, CONTAR.SI y PROMEDIO.SI. Utilicemos como agrupamiento el año y el mes. ¿Podríamos hacer esto de una manera distinta?

\subsection{Funciones para carácteres}

\begin{itemize}
	\item IZQUIERDA
	\item DERECHA
	\item EXTRAE
	\item ESTEXTO
	\item ENCONTRAR
	\item REEMPLAZAR
\end{itemize}

\subsection{Funciones para fechas}

\begin{itemize}
	\item FECHA
	\item DIAS
	\item MES
	\item MINUTO
	\item AÑO
\end{itemize}

\section{Desafío clase pasada \faWarning}

\begin{itemize}
	\item Genere una función que retorno el primer valor no blanco (vacío de un vector). 
	
	Resolvamos dicho ejercicio. Para esto, considerar las siguientes funciones:
	
	\begin{itemize}
		\item INDICE
		\item COINCIDIR
		\item ESBLANCO
	\end{itemize}

\end{itemize}

\begin{solution}
	Se debe aplicar funciones de la siguiente manera:\\
	
	=INDICE(C4:N4;1;COINCIDIR(1;INDICE(1-ESBLANCO(C4:N4);1;0);0)) \\
	
	Veamos esto parte por parte.
\end{solution} 

\section{Próxima evaluación}

\begin{itemize}
	\item Generar grupos de cuatro personas (sin excepciones).
	
	\item Fecha 27/04/2021 (3 semanas más aprox.).
	
	\item Se entregará una tarea en dicha semana, y tendrán hasta el 04/05/2021 para entregarla. 
\end{itemize}

\section{Fechas relevantes}

\begin{center}
\begin{tabular}{|c|l|c|c|}
	\hline
	Unidad & Evaluación & Ponderación & Fecha \\
	\hline
	Unidad I & Evaluación diagnóstica & & 25/03/2021 \\
	\hline
	& Evaluación Individual Participación & (5\%) & 05/04/2021 \\
	\hline
	& Evaluación Grupal & (15\%) & 27/04/2021 - 04/05/2021 \\
	\hline
	& Evaluación Individual - Sumativa I & (30\%) & 11/05/2021 \\
	\hline
	Unidad II & Evaluación Formativa & & 13/05/2021 \\
	\hline
	& Evaluación Individual Participación & (5\%) & 27/05/2021 \\
	\hline
	& Evaluación Grupal & (10\%) & 08/06/2021 - 15/06/2021 \\
	\hline
	& Evaluación Individual - Sumativa II & (15\%) & 17/06/2021 \\
	\hline
	Unidad III & Evaluación Formativa & & 22/06/2021 \\
	\hline
	& Evaluación Individual Participación & (5\%) & 24/06/2021 \\
	\hline
	& Evaluación Individual Sesión I- Sumativa III & (15\%) & 08/07/2021 \\
	\hline
	& Evaluación Individual Sesión II- Sumativa III & (15\%) & 13/07/2021 \\
	\hline
\end{tabular}
\end{center}

\end{document}



